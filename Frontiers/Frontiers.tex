%%%%%%%%%%%%%%%%%%%%%%%%%%%%%%%%%%%%%%%%%%%%%%%%%%%%%%%%%%%%%%%%%%%%%%%%%%%%%%%%%%%%%%%%%%%%%%%%%%%%%%%%%%%%%%%%%%%%%%%%%%%%%%%%%%%%%%%%%%%%%%%%%%%%%%%%%%%
% This is just an example/guide for you to refer to when submitting manuscripts to Frontiers, it is not mandatory to use Frontiers .cls files nor frontiers.tex  %
% This will only generate the Manuscript, the final article will be typeset by Frontiers after acceptance.
%                                              %
%                                                                                                                                                         %
% When submitting your files, remember to upload this *tex file, the pdf generated with it, the *bib file (if bibliography is not within the *tex) and all the figures.
%%%%%%%%%%%%%%%%%%%%%%%%%%%%%%%%%%%%%%%%%%%%%%%%%%%%%%%%%%%%%%%%%%%%%%%%%%%%%%%%%%%%%%%%%%%%%%%%%%%%%%%%%%%%%%%%%%%%%%%%%%%%%%%%%%%%%%%%%%%%%%%%%%%%%%%%%%%

%%% Version 3.4 Generated 2018/06/15 %%%
%%% You will need to have the following packages installed: datetime, fmtcount, etoolbox, fcprefix, which are normally inlcuded in WinEdt. %%%
%%% In http://www.ctan.org/ you can find the packages and how to install them, if necessary. %%%

\documentclass[utf8]{frontiersSCNS}

%\setcitestyle{square} % for Physics and Applied Mathematics and Statistics articles
\usepackage{url,hyperref,lineno,microtype,subcaption}
\usepackage[onehalfspacing]{setspace}

\linenumbers


% BELOW TAKEN FROM rticles plos template
%
% amsmath package, useful for mathematical formulas
\usepackage{amsmath}
% amssymb package, useful for mathematical symbols
\usepackage{amssymb}

% hyperref package, useful for hyperlinks
\usepackage{hyperref}

% graphicx package, useful for including eps and pdf graphics
% include graphics with the command \includegraphics
\usepackage{graphicx}

% Sweave(-like)
\usepackage{fancyvrb}
\DefineVerbatimEnvironment{Sinput}{Verbatim}{fontshape=sl}
\DefineVerbatimEnvironment{Soutput}{Verbatim}{}
\DefineVerbatimEnvironment{Scode}{Verbatim}{fontshape=sl}
\newenvironment{Schunk}{}{}
\DefineVerbatimEnvironment{Code}{Verbatim}{}
\DefineVerbatimEnvironment{CodeInput}{Verbatim}{fontshape=sl}
\DefineVerbatimEnvironment{CodeOutput}{Verbatim}{}
\newenvironment{CodeChunk}{}{}

% cite package, to clean up citations in the main text. Do not remove.
\usepackage{cite}

\usepackage{color}

\providecommand{\tightlist}{%
  \setlength{\itemsep}{0pt}\setlength{\parskip}{0pt}}

% Below is from frontiers
%
\bibliographystyle{frontiersinSCNS}
% Use doublespacing - comment out for single spacing
%\usepackage{setspace}
%\doublespacing


% Leave a blank line between paragraphs instead of using \\


\def\keyFont{\fontsize{8}{11}\helveticabold }


%% ** EDIT HERE **
%% PLEASE INCLUDE ALL MACROS BELOW

%% END MACROS SECTION

% Pandoc citation processing
\newlength{\csllabelwidth}
\setlength{\csllabelwidth}{3em}
\newlength{\cslhangindent}
\setlength{\cslhangindent}{1.5em}
% for Pandoc 2.8 to 2.10.1
\newenvironment{cslreferences}%
  {}%
  {\par}
% For Pandoc 2.11+
\newenvironment{CSLReferences}[2] % #1 hanging-ident, #2 entry spacing
 {% don't indent paragraphs
  \setlength{\parindent}{0pt}
  % turn on hanging indent if param 1 is 1
  \ifodd #1 \everypar{\setlength{\hangindent}{\cslhangindent}}\ignorespaces\fi
  % set entry spacing
  \ifnum #2 > 0
  \setlength{\parskip}{#2\baselineskip}
  \fi
 }%
 {}
\usepackage{calc} % for calculating minipage widths
\newcommand{\CSLBlock}[1]{#1\hfill\break}
\newcommand{\CSLLeftMargin}[1]{\parbox[t]{\csllabelwidth}{#1}}
\newcommand{\CSLRightInline}[1]{\parbox[t]{\linewidth - \csllabelwidth}{#1}\break}
\newcommand{\CSLIndent}[1]{\hspace{\cslhangindent}#1}
\usepackage{booktabs}
\usepackage{longtable}
\usepackage{array}
\usepackage{multirow}
\usepackage{wrapfig}
\usepackage{float}
\usepackage{colortbl}
\usepackage{pdflscape}
\usepackage{tabu}
\usepackage{threeparttable}
\usepackage{threeparttablex}
\usepackage[normalem]{ulem}
\usepackage{makecell}
\usepackage{xcolor}


\def\Authors{
  Maxwel C Oliveira\,\textsuperscript{1},
  Amit J Jhala\,\textsuperscript{2},
  Mark Bernards\,\textsuperscript{3},
  Chris Proctor\,\textsuperscript{2},
  Strahinja Stepanovic\,\textsuperscript{2},
  Rodrigo Werle\,\textsuperscript{1*}}

\def\Address{

  \textsuperscript{1} Department of Agronomy, University of
Wisconsin-Madison,  Madison,  Wisconsin,  United States
  
  \textsuperscript{2} Department of Agronomy and
Horticulture, University of
Nebraska-Lincoln,  Lincoln,  Nebraska,  United States
  
  \textsuperscript{3} Department of Agronomy, Western Illinois
University,  Macomb,  Illnois,  United States
  }

  
  \def\firstAuthorLast{Oliveira {et~al.}}
  
  
  
  
  
  
  
  
  \def\corrAuthor{Rodrigo Werle}\def\corrAddress{Department of Agronomy,
University of Wisconsin-Madison, United States\\1575 Linden
Dr\\Madison, Wisconsin, 53705 United
States}\def\corrEmail{\href{mailto:rwerle@uwisc.edu}{\nolinkurl{rwerle@uwisc.edu}}}
  


\begin{document}
\onecolumn
\firstpage{1}

\title[Palmer amaranth adaptation]{Palmer amaranth (\emph{Amaranthus
palmeri}) adaptation to US Midwest agroecosystems}
\author[\firstAuthorLast]{\Authors}
\address{} %This field will be automatically populated
\correspondance{} %This field will be automatically populated

\extraAuth{}% If there are more than 1 corresponding author, comment this line and uncomment the next one.
%\extraAuth{corresponding Author2 \\ Laboratory X2, Institute X2, Department X2, Organization X2, Street X2, City X2 , State XX2 (only USA, Canada and Australia), Zip Code2, X2 Country X2, email2@uni2.edu}


\maketitle

\begin{abstract}

Abstract length and content varies depending on article type. Refer to 
\url{http://www.frontiersin.org/about/AuthorGuidelines} for abstract requirement
and length according to article type.

%All article types: you may provide up to 8 keywords; at least 5 are mandatory.
\tiny
 \keyFont{ \section{Keywords:} Evolution Flowering Management Pigweed Weed} 

\end{abstract}

\hypertarget{introduction}{%
\section*{Introduction}\label{introduction}}
\addcontentsline{toc}{section}{Introduction}

Palmer amaranth (\emph{Amaranthus palmeri} S. Watson) is currently
ranked as one of the most economically detrimental weed species to
cropping systems in the United States (Van Wychen, 2020). Unmanaged
Palmer amaranth plants compete for water, light, and nutrients, which
can drastically impact crop yields (Berger et al., 2015). For example,
Palmer amaranth has been documented to reduce up to 91\%, 68\%, and 54\%
corn (Massinga et al., 2001), soybean (Klingaman and Oliver, 1994), and
cotton (Morgan et al., 2001) yields, respectively. Moreover, Palmer
amaranth has shown a remarkable capacity to evolve resistance to
herbicides. To date, Palmer amaranth has evolved resistance to eight
herbicide sites of action (Heap, 2021), increasing the weed management
complexity (Lindsay et al., 2017). Thus, Palmer amaranth poses an
economical and environmental risk to sustainable agriculture.

Palmer amaranth is a fast growing summer annual forb indigenous to the
Sonoran Desert (Sauer, 1957). The species would eventually emerge as a
threat to US agriculture in the 1990s. Palmer amaranth weediness is
likely a result of human-assisted selection in combination with plant
biology. Farm mechanization, adoption of conservation agriculture (e.g.,
no-till), and reliance on herbicides for weed management are the main
human-mediated selection of Palmer amaranth into cropping systems (Ward
et al., 2013). On the other hand, Palmer amaranth is a prolific seed
producer with a C4 photosynthetic apparatus (Wang et al., 1992). With a
dioecy nature, Palmer amaranth male and female plants are obligate
outcrosser species, increasing the chances of exchanging adaptive traits
among plants (Oliveira et al., 2018). Also, Palmer amaranth small seed
size (e.g, 1 mm) tend to thrive in no-tillage systems (Price et al.,
2011), and spread across locations through farm equipment (Sauer, 1972),
manure (Hartzler and Anderson, 2016), animals (Farmer et al., 2017). The
dispersal capacity of Palmer amaranth makes the species one of the most
successful cases of weed adaptation to current cropping systems.

Light and temperature are the main environment requirements for Palmer
amaranth successful grow and development (Jha et al., 2010). Palmer
amaranth is reported with an extended germination period (Ward et al.,
2013). Germination of Palmer amaranth was triggered by 18 C soil
temperature at 5 cm depth (Keeley et al., 1987), and optimal germination
and biomass production occurred at 35/30 C day and night temperatures
(Guo and Al-Khatib, 2003). In addition, Palmer amaranth establishment is
human-mediated by tillage timings and preemergence-applied herbicides
(Chahal et al., 2021), which can result in weed germination shifts
(Sbatella and Wilson, 2010). Water has not shown to limit Palmer
amaranth fitness. Under continuous water stress, Palmer amaranth
survived and produced at least 14000 seeds plant-1 (Chahal et al.,
2018). Seeds from Palmer amaranth growing with limited water conditions
were heavier, less dormant, and prompt for germination (Matzrafi et al.,
2021). Growing conditions and management practices also influence Palmer
amaranth sex dimorphism and flowering pattern (Korres et al., 2017;
Rumpa et al., 2019). Therefore, Palmer amaranth has shown plasticity to
evolve and fast adapt under selection pressure. Palmer amaranth
invasion/adaptation into new habitats is likely to increase with the
global temperature warming. Currently, it is estimated that the greatest
climatic risk of Palmer amaranth establishment are agronomic crops in
Australia and Sub-Sahara Africa (Kistner and Hatfield, 2018).
Temperature is a key factor limiting Palmer amaranth expansion to cooler
geographies (Briscoe Runquist et al., 2019); however, under future
climate change Palmer amaranth is likely to expand northward into Canada
and Northern Europe (Kistner and Hatfield, 2018; Briscoe Runquist et
al., 2019).

Palmer amaranth is already found in agronomic crops of South America
(Larran et al., 2017; Küpper et al., 2017) and Southern Europe (Milani
et al., 2021). In the United States, Palmer amaranth is well established
in the Cotton Belt (Garetson et al., 2019; Bagavathiannan and
Norsworthy, 2016) in the southern United States but its range is
expanding northward. For example, herbicide resistant Palmer amaranth is
widespread in Nebraska (Oliveira et al., 2021). There are some reported
cases of Palmer amaranth in Michigan (Kohrt et al., 2017) and
Connecticut (Aulakh et al., 2021). Also, it is estimated that Palmer
amaranth can cause high damage to soybean fields in Illinois (Davis et
al., 2015), which is concerning as soybean along with corn make most of
US Midwest agronomic hectares. In Iowa, a study showed that Palmer
amaranth is still not well adapted compared to waterhemp
(\emph{Amaranthus tuberculatus}) (Baker, 2021). Invasion and successful
eradication of Palmer amaranth is documented in Minnesota (Yu et al.,
2021). Palmer amaranth infestations have not been detected in Canada;
however, Palmer amaranth seeds were detected in sweet potato slips in
the country (Page et al., 2021). Palmer amaranth is still not as well
adapted and established to Northern as it is in the Southern North
America. Therefore, its range of expansion into new habitats is
increasing. It seems fated the need to manage Palmer amaranth in
agronomic crops throughout multiple environments in the near future.
Strategies on Palmer amaranth management should encompass the
agroecosystem level but not only attempts to eradicate the weed. Most
tactics to manage Palmer amaranth are based on technology fixes (Scott,
2011), which are short-term (e.g., herbicide and/or tillage) rather than
long-term weed management. Palmer amaranth management should be built on
minimizing the species ability to adapt, grow and develop into
agroecossystems.

In the southeastern US, early growing Palmer amaranth is well known to
have a higher impact on cotton yields compared to late established
plants (MacRae et al., 2013). In the northern states, Palmer amaranth
impact on the agroecosystem is recent. Studies investigating Palmer
amaranth in those locations are limited due to the plant classification
as noxious weed species in some northern states (Yu et al., 2021).
Nonetheless, the continuous Palmer amaranth dispersal and potential
establishment across the northern United States is concerning and
warrants investigations on species morphology in such environments.
Understanding Palmer amaranth biology and growing strategies under
different agroecosystems can enhance our knowledge on species adaptation
and management practices. It can also aid in designing proactive and
ecological tactics to limit the species range expansion, reduce its
negative impact, and developing resilient and sustainable farming
systems (MacLaren et al., 2020). Therefore, the objective of this study
was to investigate the flowering pattern, gender, biomass production,
and height of Palmer amaranth cohorts growing under corn, soybean and
bareground across five locations in the United States Midwest.

\hypertarget{results}{%
\section*{Results}\label{results}}
\addcontentsline{toc}{section}{Results}

\hypertarget{plant-material-and-growing-conditions}{%
\subsection*{Plant material and growing
conditions}\label{plant-material-and-growing-conditions}}
\addcontentsline{toc}{subsection}{Plant material and growing conditions}

A Palmer amaranth accession (Per1) from Perkins County, Nebraska. with
no reported herbicide resistance according to Oliveira et al. (2021) was
selected for this study. Three weeks prior to the onset of the field
experiments, seeds were planted in plastic trays containing potting-mix.
Emerged seedlings (1 cm) were transplanted into 200 cm-3 plastic pots (a
plant pot-1). Palmer amaranth seedlings were supplied with adequate
water and kept under greenhouse conditions at the University of
Wisconsin-Madison, University of Nebraska-Lincoln, and Western Illinois
University; and kept outdoors in the Perkins extension office at Grant,
NE until the 2-3 leaf stage (5 to 8 cm height) when they were
transported to the field.

\hypertarget{field-study}{%
\subsection*{Field study}\label{field-study}}
\addcontentsline{toc}{subsection}{Field study}

The experiment was conducted in 2018 and 2019 under field conditions at
five locations: Arlington, WI (43°18'N, 89°29'W), Clay Center, NE ('N,
'W), Grant, NE (('N, 'W)), Lincoln, NE (('N, 'W)), and Macomb, IL (('N,
'W)).

Fields were conventionally tilled prior to crop planting. Corn hybrid
and soybean varieties were planted in 76-cm row spacing (Table 1).
Monthly mean air temperature and total precipitation were obtained using
Daymet weather data from June through September across the five
locations in 2018 and 2019 (Correndo et al., 2021) (Figure 1)

\begin{table}[!h]

\caption{\label{tab:unnamed-chunk-2}Field study attributes }
\centering
\resizebox{\linewidth}{!}{
\fontsize{10}{12}\selectfont
\begin{tabular}[t]{lllllll}
\toprule
Attributes &  & Arlington, WI & Clay Center, NE & Grant, NE & Lincoln, NE & Macomb, IL\\
\midrule
Bareground &  &  &  &  &  & \\

 & Weed control & glyphosate & saflufenacil + imazethapyr + pyroxasulfone &  &  & \\

Corn &  &  &  &  &  & \\

 & Hybrid &  & DKC60-67 &  &  & \\

 & Seeding rate &  & 86487 &  &  & \\

 & Weed control & glyphosate / S-metolachor & S-metolachlor + trazine + mesotrione, + bicyclopyrone &  &  & \\

 & Stage 1 cohort & V2-3 &  &  &  & \\

\multirow{-5}{*}{\raggedright\arraybackslash } & Stage 2 cohort & V6-7 &  &  &  & \\

Soybean &  &  &  &  &  & \\

 & Variety & DSR-1950 & AG21X8 &  &  & \\

 & Seeding rate & 296400 & 321237 &  &  & \\

 & Stage 1 cohort & V1-2 &  &  &  & \\

 & Stage 2 cohort & V5-6 &  &  &  & \\

\multirow{-5}{*}{\raggedright\arraybackslash } & Weed control & glyphosate / S-metolachor & saflufenacil + imazethapyr + pyroxasulfone &  &  & \\

Planting day &  & May 10 to 20 & May 10 to 14 &  &  & \\

Soil &  &  &  &  &  & \\

 & Type &  & Crete Silt Loam &  &  & \\

 & Ratio (sand-clay-silt) &  & 58-25-6.5 &  &  & \\

 & pH & 6.6 & 6.5 &  &  & \\

\multirow{-4}{*}{\raggedright\arraybackslash } & Organic matter (\%) &  & 3 &  &  & \\
\bottomrule
\end{tabular}}
\end{table}

The field experimental units were three adjacent 9.1 m wide (12 rows at
76.2 cm row spacing) by 10.7 m long. The experimental design were
arranged in factorial design with three crops, two transplanting times
simulating two cohorts, repeated across five locations. Each
experimental unit was planted with corn, soybean, or kept under
bareground. The two transplanting timings were June 1 (first cohort) and
July 1 (second cohort). Palmer amaranth seedlings (potting mix + two
seedlings) were transplanted (6 cm deep and 8 cm wide). Forty-eight
plants were equidistantly placed (0.76 m apart) between rows within each
crop (Figure 2). After a week, one plant was eliminated and one was
kept, resulting in 24 plants per experimental unit and transplanting
time (Figure 2). When needed, Palmer amaranth plants were supplied with
water during the first week after transplanting.

After transplanting, Palmer amaranth flowering was monitored until the
end of the study. When a plant flowered, the day was recorded, plant sex
was identified (male or female), plant height was measured from soil
surface to the top of plant. Also, aboveground plant biomass was harvest
near soil surface and oven dried at 65 C until reaching constant weight
before weighing (g plant-1 was recorded).

Plants had to be harvested at flowering because Palmer amaranth is
neither endemic in Wisconsin nor in Illinois. In our study, all
locations follow the methodology of plant harvest at flowering
initiation, except in Grant, NE. In this location, all Palmer amaranth
plants were harvest at once on July 06, 2018 and 2019 (first cohort),
and on August 17, 2018 and on July 31, 2019 (second cohort).

\hypertarget{statistical-analyses}{%
\subsection*{Statistical analyses}\label{statistical-analyses}}
\addcontentsline{toc}{subsection}{Statistical analyses}

The statistical analyses were performed using R statistical software
version 4.0.1.

Analyses of Palmer amaranth height and biomass were performed with a
linear mixed model using \emph{lmer} function from ``lme4'' package
(Bates et al., 2015). Plant height and biomass were log transformed to
meet model assumption of normality. In the model, crop (bareground,
corn, soybean) was the fixed effect and year nested with location the
random effects. Analysis of variance was performed with \emph{anova}
function from ``car'' package (Fox and Weisberg, 2018). Marginal means
and compact letter display were estimated with \emph{emmeans} and
\emph{cld} from packages ``emmeans'' and ``multcomp'' (Hothorn et al.,
2008).

The Palmer amaranth flowering timing was estimated as cumulative
flowering across all location. Dataset was nested but id did not include
Grant, NE. Palmer amaranth cumulative flowering estimation was
determined using an asymmetrical three parameter log logistic Weibull
model of the drc package (Ritz et al., 2015).

\[Y(x) = 0 + (d-0) exp (-exp(b(log(x)-e)))\]

In this model, \emph{Y} is the Palmer amaranth cumulative flowering,
\emph{d} is the upper limit (set to 100), and \emph{e} is the XXX, and
\emph{x} day of year (doy).

The doy for 10, 50, and 90\% Palmer amaranth cumulative flowering were
determined using the \emph{ED} function of drc package. Also, the 10,
50, and 90\% Palmer amaranth cumulative flowering were compared among
crops and cohorts using the \emph{EDcomp} function of drc package. The
EDcomp function compares the ratio of cumulative flowering using
t-statistics, where P-value \textless{} 0.05 indicates that we fail to
reject the null hypothesis.

Palmer amaranth sex was fitted to a binary logistic regression
(Bangdiwala, 2018). Binary logistic regression is used for predicting
binary classes, such as the probability of a plant being female in a
dioecious species. Prior to the analysis, all missing values were
removed from the dataset. Also, data from Grant was not used in this
analysis due to the uniform plant harvesting at that location. The
complete dataset was splitted into 80\% train and 20\% test data. The
80\% train is used for the model training and the 20\% test is used for
checking how the model generalized on unseen dataset. With 80\% dataset,
a binary response variable, male (0) and female (1), was fitted to a
generalized linear model (\emph{glm} function) including day of year
harvest, height, weight, crop and month as independent variables
(without interaction). The model family was binomial with a logit
function. The model fit was assessed through pseudo R-squared values
(McFadden, Cox and Snell, Cragg and Uhler) and likelihood ratio using
\emph{nagelkerke} function (``rcompanion'' package). The marginal
effects computation was performed with Average Marginal Effects (AMEs)
at every observed value of x and average across the results (Leeper,
2017) using \emph{margins} function from ``margins'' package. The rest
20\% dataset was predicted using \emph{predict} function with a cutoff
estimation for male or female using \emph{performance} function. The
model quality prediction from the classification algorithm was measured
with precision (\emph{precision} function), recall (\emph{recall}
function) and F1-score (\emph{f\_meas} function) using the ``yardstick''
package. The precision determines the accuracy of positive predictions
(female plants), recall determines the fraction of positives that were
correctly identified, and F1-score is a weighted harmonic mean of
precision and recall with the best score of 1 and the worst score of 0.
F1-score conveys the balance between the precision and the recall
(Yacouby and Axman, 2020).

\hypertarget{results-1}{%
\section*{Results}\label{results-1}}
\addcontentsline{toc}{section}{Results}

\hypertarget{palmer-amaranth-height-and-biomass}{%
\subsection*{Palmer amaranth height and
biomass}\label{palmer-amaranth-height-and-biomass}}
\addcontentsline{toc}{subsection}{Palmer amaranth height and biomass}

Palmer amaranth plants accumulated more biomass when growing in
bareground compared to plants growing in soybean and corn (figure 3A).
Palmer amaranth plants in the first cohort produced 75.5, 28.3 g, and
16.3 g plant-1 in bareground, soybean and corn, respectively. Plants
from the second cohort produced 62.6 g plant in bareground, followed by
6.3 g plant in soybean, and 1.4 g plant in corn.

Palmer amaranth height was more uniform across cohort timings, except
when growing in corn (figure 3B). Palmer amaranth plants from the first
cohort were on average 69.2 cm tall in bareground, which was not
different from the 70.7 cm tall plants from the second cohort timing (P
= 0.74). In addition, no difference in Palmer amaranth height (69.3 cm)
was detected from first cohort plants in soybean to first and second
cohort plants in bareground (P \textgreater{} 0.75). Palmer amaranth
plants from the second cohort were nearly 10 cm lower compared to the
first cohort in soybeans (P = 0.04). The tallest (first cohort) and
smallest (second cohort) Palmer amaranth plants were found in corn.
Palmer amaranth reached 85.2 and 38.2 cm tall, respectively.

\hypertarget{palmer-amaranth-cumulative-flowering}{%
\subsection*{Palmer amaranth cumulative
flowering}\label{palmer-amaranth-cumulative-flowering}}
\addcontentsline{toc}{subsection}{Palmer amaranth cumulative flowering}

Palmer amaranth plants from the first cohort growing in corn resulted in
a longer flowering window compared to plants growing in bareground and
soybean (Figure 4A). The 10\% cumulative Palmer amaranth flowering in
soybean, bareground and corn occurred at the end of June. Palmer
amaranth reached 10\% flowering in soybean, bareground and corn at doy
180, 180.9 and 181.7, respectively. The 50\% Palmer amaranth cumulative
flowering occurred in July. Palmer amaranth reached 50\% flowering in
bareground, soybean and corn at doy 193.4, 194.8, and 206.6,
respectively. Similar trend was observed at 90\% Palmer amaranth
cumulative flowering. Palmer amaranth reached 90\% flowering at doy
252.6 in corn (early September), which was 38 and 32 days after reaching
90\% flowering in bareground and soybean, respectively.

Palmer amaranth cumulative flowering at the second cohort ranged from
mid July to mid September (Figure 4B). Palmer amaranth growing in the
bareground resulted in earlier flowering time compared to soybean and
corn. Palmer amaranth growing in bareground reached 10\%, 50\%, and 90\%
flowering time at day 203.8, 214.4, and 232.2, respectively. Palmer
amaranth growing in soybean reached 10\% flowering at doy 210.9, which
was 6 days prior to corn (\emph{P}-value = 0.00). Similar trend was
observed at 50\% flowering, whereas Palmer amaranth reached 50\%
flowering in corn (doy 233.0) 4 days after soybeans (doy 228.9; \emph{P}
= 0.00). The 90\% Palmer amaranth cumulative flowering occurred at same
day in corn (260.9) and soybean (260.5; \emph{P} = 0.66).

\hypertarget{palmer-amaranth-gender}{%
\subsection*{Palmer amaranth gender}\label{palmer-amaranth-gender}}
\addcontentsline{toc}{subsection}{Palmer amaranth gender}

The model fit was 0.23, 0.32, 0.40 using pseudo R-squared test from
McFadden, Cox and Snell, and Cragg and Uhler, respectively. The
likelihood ratio test showed a p-value of \textless{} 0.00. The average
marginal effects showed that Palmer amaranth growing in corn resulted in
14.8\% less females plants (Table 2). Moreover, increasing a unit doy
increases the probability of having a female plant by 0.4\% (Table 2 and
Figure4A). Similar trend is observed for weight as well as height,
whereas the probability of being female increase by 0.2\% (Figure 4B)
and 0.1\% (Figure 4C) when a unit of weight (g) and height (cm)
increases, respectively.

\begin{table}[!h]

\caption{\label{tab:unnamed-chunk-3}Average marginal means of gender logistic model. Factor pararemter values (e.g. crop) is shown related to soybean.}
\centering
\fontsize{10}{12}\selectfont
\begin{tabular}[t]{lrrrrrr}
\toprule
Term & AME & SE & Lower & Upper & Z-score & P-value\\
\midrule
crop\_bareground & -0.048 & 0.054 & -0.154 & 0.059 & -0.876 & 0.381\\

crop\_corn & -0.148 & 0.052 & -0.250 & -0.046 & -2.842 & 0.004\\

doyh & 0.004 & 0.001 & 0.003 & 0.006 & 4.959 & 0.000\\

height & 0.002 & 0.001 & 0.001 & 0.003 & 2.953 & 0.003\\

weight & 0.001 & 0.000 & 0.000 & 0.001 & 2.179 & 0.029\\
\bottomrule
\multicolumn{7}{l}{\rule{0pt}{1em}\textsuperscript{a} Average Marginal Effects. \textsuperscript{b} Standard Error.}\\
\end{tabular}
\end{table}

The model accuracy evaluation accuracy in the 20\% test dataset was 0.62
with a cutoff value for female and male plants of 0.43. The model
classification showed a precision of 0.64, recall of 0.66, and a F means
score of 0.65. In addition, the area under the curve was 0.64.

\hypertarget{discussion}{%
\section*{Discussion}\label{discussion}}
\addcontentsline{toc}{section}{Discussion}

Our study showed that Palmer amaranth biomass, height, flowering pattern
and gender varied within agroecossystems and cohort timings. In general,
first cohort of Palmer amaranth plants were heavier and taller when
compared to the second cohort. At first cohort, resources (e.g., soil
nutrients) and conditions (e.g., light) were more timely available for
the species. High biomass and taller Palmer amaranth plants are likely a
weed strategy to compete for light in between crop rows in absence of
canopy. In such conditions, Palmer amaranth showed an extraordinary
plasticity to adapt upon the agroecosystem. This is evident when
comparing Palmer amaranth canopy shape, and its extended flowering
pattern when growing into corn compared to soybean. The Palmer amaranth
competition strategy was to mimic the crop grow and development (Figure
6). These results suggests that Palmer amaranth can fast evolve
life-history traits to adapt into cropping systems and cultural
practices, which was also showed in a study varying nitrogen
fertilization (Bravo et al., 2018). Our results highlight the Palmer
amaranth as a threat to field crops as breeding more competitive crop
varieties is likely to select more competitive biotypes (Bravo et al.,
2017).

Palmer amaranth grow and development in second cohort was limited due to
the crop competitive ability at advanced development stages. Palmer
amaranth was transplanted when corn canopy was nearly closed, which
reduced Palmer amaranth competitiveness. As a result, Palmer amaranth
height and biomass was lower compared to its first cohort. Under crop
canopy (e.g., second cohort), Palmer amaranth flowering window was near
to similar in corn and soybean. Palmer amaranth growing without crop
competition produced the highest amounts of biomass and less extended
flowering window. The Palmer amaranth strategy in bareground was to
invest biomass in growing plant width and height. Nonetheless, Palmer
amaranth produced 21\% less biomass in second cohort compared to first
cohort timing. In a bareground study, early emerged Palmer amaranth
without competition was 50\% taller than late emerged plants (Webster
and Grey, 2015). These results suggest that crop competition is not the
only factor limiting late Palmer amaranth establishment. The limited
growth of Palmer amaranth at second cohort is likely a reduced plant
response to day length, light availability and thermal units (e.g,
growing degree days). It is hypothesize that reduced day length
contributed to smaller plants at second cohort as well as shorter
flowering period. A study in North Carolina and Illinois predicted that
less than 10\% Palmer amaranth seedlings emergence occurred after June
(Piskackova et al., 2021). In addition, Palmer amaranth negative impact
on soybean (Korres et al., 2020) and cotton (Webster and Grey, 2015)
yields was higher when plants were established near to crop planting.

Seed production was not evaluated due to plant harvest at initiation of
flowering. Nonetheless, it is well documented a strong positive
correlation between Palmer amaranth biomass and seed production
(Schwartz et al., 2016; Spaunhorst et al., 2018). I our study, Palmer
amaranth growing at first cohort accumulated an overall 36\% more
biomass when compared to second cohort. Therefore, Palmer amaranth
plants growing in the second cohort is likely to produce less seeds
regardless the cropping system. Our observation is consistent with the
findings that first Palmer amaranth cohort produced 50\% more seeds per
plant than Palmer amaranth plants established six weeks later in
bareground (Webster and Grey, 2015). Still, seed production at second
cohort is likely to replenish the soil seedbank. Seed production and
deposition in the seedbank is also a key factor for species perpetuation
(Menges, 1987). Palmer amaranth can produce hundred thousands seeds per
plant (Schwartz et al., 2016; Keeley et al., 1987), and stay viable
buried in the seedbank for at least 36 months (Sosnoskie et al., 2013).
Therefore, preventing Palmer amaranth seed production or/and seed
migration to its non-native habitat is an essential strategy to minimize
weed impact into agroecossystem (Davis et al., 2015).

An ecological approach to reduce seed production in Palmer amaranth is
understanding its flowering window. Our study suggests that Palmer
amaranth flowering was slightly influenced by cropping systems and
cohort timings. Palmer amaranth growing in bareground and corn resulted
in the overall shortest and longest flowering window, respectively. When
growing in soybean, Palmer amaranth flowering window was similar to
bareground at first cohort but similar to corn at second cohort timing.
Plant flowering initiation is complex and depends on biological and
ecological factors (Lang, 1965). We hypothesize that when growing in
high competition (e.g., second cohort), Palmer amaranth plants tend to
initiate flowering early, as well as having an extended flowering
window. Early flower initiation is plant strategy when growing in stress
conditions. For example, when growing under water stress, early
flowering in Palmer amaranth resulted in a mismatch between female and
male plants by seven days (Mesgaran et al., 2021). A mismatch in Palmer
amaranth male and female flowering period can minimize plant outcross,
and thus reduce plant seed production and exchange of resistant alleles
(Jhala et al., 2021). Sex dimorphism manipulation is considered a
potential ecological pest control (McFarlane et al., 2018; Schliekelman
et al., 2005).

The mechanisms of sex-determination in plant species is intriguing and
arouse the curiosity of many scientists, including Charles Darwin
(Darwin, 1888). In our study, the gender model performance was decent
(AIC 0.64) considering the biology of plant flowering. A 1:1 male and
female sex ratio is a general evolutionary stable strategy for plant
species perpetuation (Fisher, 1930). However, a slight deviation from
1:1 sex ratio might occur in some dioecious species. For example, the
dioecious \emph{Halophila stipulacea} is a female-biased plant in its
native habitat, but the naturalized \emph{H. stipulacea} have a 1:1
ratio (Nguyen et al., 2018). Naturalized of \emph{H. stipulacea} reduced
female-male ratio to expand into its non-native habitat (Nguyen et al.,
2018). Also, biotic and/or abiotic stress can influence plant sex
determination. Palmer amaranth male-to-female ratio was greater under
high plant densities (Korres and Norsworthy, 2017) and after herbicide
application (Rumpa et al., 2019). Our model estimated that late
flowering, heavier and taller Palmer amaranth plants deviated from 1:1
ratio in favor to female plants. It was reported that female Palmer
amaranth plants invested more in height, stem and biomass while male
invested more in leaf area and leaf dry weight under nutrient deficiency
(Korres et al., 2017). Our model also estimated more female plants in
soybean and bareground compared to corn, which might linked to plant
competition strategy in each agroecossystem. Our results showed the
influence of life-history and ecological traits on sexual dimorphism in
Palmer amaranth. Sexual dimorphism is documented in other dioecious
species (Barrett and Hough, 2013). For example, stronger female plant
competition and greater male tolerance to herbivory was reported in
\emph{Spinacia oleracea} (Pérez-Llorca and Sánchez Vilas, 2019).
Research on candidate genes for sex determination in \emph{Amaranthus}
species are currently underway but it is far to complete (Montgomery et
al., 2021, 2019). Further studies are also needed to understand the
ecological basis of Palmer amaranth flowering, including the plant
behavior under climate change.

Our study demonstrated the short-term Palmer amaranth plasticity to grow
and develop into cropping-systems. Is likely that Palmer amaranth range
will continue to expanding to new geographies. Therefore, preventive
management is a priority to minimizing Palmer amaranth dispersal.
Reactive management should focus on early-season management programs,
which would have a large negative effect on Palmer amaranth growth and
development. Tactics that promote early-season crop advantage against
Palmer amaranth, including early crop planting, crop rotation (Oliveira
et al., 2021), plant width, preemergence applied herbicide (Sanctis et
al., 2021), and crop residue (e.g.~cover crops) would minimize the
negative impact of Palmer amaranth in agroecossystems. The
aggressiveness and differential Palmer amaranth adaptation to
agroecosystem is striking and require national efforts to minimize the
species impact on economy and sustainability.

\hypertarget{disclosureconflict-of-interest-statement}{%
\section*{Disclosure/Conflict-of-Interest
Statement}\label{disclosureconflict-of-interest-statement}}
\addcontentsline{toc}{section}{Disclosure/Conflict-of-Interest
Statement}

The authors declare that the research was conducted in the absence of
any commercial or financial relationships that could be construed as a
potential conflict of interest.

\hypertarget{author-contributions}{%
\section*{Author Contributions}\label{author-contributions}}
\addcontentsline{toc}{section}{Author Contributions}

RW and MO: designed the experiments; AJ, CP, MB, MO, and SS: conducted
the experiments; MO: analyzed the data and wrote the manuscript; AJ, CP,
MB, MO, SS, and RW: conceptualized the research. All authors reviewed
the manuscript.

\hypertarget{acknowledgments}{%
\section*{Acknowledgments}\label{acknowledgments}}
\addcontentsline{toc}{section}{Acknowledgments}

Funding: This work received no specific grant from any funding agency,
commercial, or not-for-profit sectors

\hypertarget{supplemental-data}{%
\section{Supplemental Data}\label{supplemental-data}}

Supplementary Material should be uploaded separately on submission, if
there are Supplementary Figures, please include the caption in the same
file as the figure. LaTeX Supplementary Material templates can be found
in the Frontiers LaTeX folder

\hypertarget{references}{%
\section{References}\label{references}}

A Frontier article expect the reference list to be included in this
section. To make that happens, the below syntax can be used. This
\href{https://pandoc.org/MANUAL.html\#placement-of-the-bibliography}{feature
is from Pandoc citeproc} which is used with \texttt{frontier\_article()}
to handle the bibliography

\hypertarget{refs}{}
\begin{CSLReferences}{1}{0}
\leavevmode\hypertarget{ref-aulakh2021}{}%
Aulakh, J. S., Chahal, P. S., Kumar, V., Price, A. J., and Guillard, K.
(2021). Multiple herbicide-resistant {Palmer} amaranth ({Amaranthus}
palmeri) in {Connecticut}: Confirmation and response to {POST}
herbicides. \emph{Weed Technology} 35, 457--463.
doi:\href{https://doi.org/10.1017/wet.2021.6}{10.1017/wet.2021.6}.

\leavevmode\hypertarget{ref-bagavathiannan2016}{}%
Bagavathiannan, M. V., and Norsworthy, J. K. (2016). Multiple-{Herbicide
Resistance Is Widespread} in {Roadside Palmer Amaranth Populations}.
\emph{PLOS ONE} 11, e0148748.
doi:\href{https://doi.org/10.1371/journal.pone.0148748}{10.1371/journal.pone.0148748}.

\leavevmode\hypertarget{ref-baker2021}{}%
Baker, R. (2021). Comparative analysis of {Palmer} amaranth
({Amaranthus} palmeri) and waterhemp ({A}. Tuberculatus) in {Iowa}.
doi:\href{https://doi.org/10.31274/etd-20210609-11}{10.31274/etd-20210609-11}.

\leavevmode\hypertarget{ref-bangdiwala2018}{}%
Bangdiwala, S. I. (2018). Regression: Binary logistic.
\emph{International Journal of Injury Control and Safety Promotion} 25,
336--338.
doi:\href{https://doi.org/10.1080/17457300.2018.1486503}{10.1080/17457300.2018.1486503}.

\leavevmode\hypertarget{ref-barrett2013}{}%
Barrett, S. C. H., and Hough, J. (2013). Sexual dimorphism in flowering
plants. \emph{Journal of Experimental Botany} 64, 67--82.
doi:\href{https://doi.org/10.1093/jxb/ers308}{10.1093/jxb/ers308}.

\leavevmode\hypertarget{ref-bates2015}{}%
Bates, D., Mächler, M., Bolker, B., and Walker, S. (2015). Fitting
{Linear Mixed}-{Effects Models Using} Lme4. \emph{Journal of Statistical
Software} 67, 1--48.
doi:\href{https://doi.org/10.18637/jss.v067.i01}{10.18637/jss.v067.i01}.

\leavevmode\hypertarget{ref-berger2015}{}%
Berger, S. T., Ferrell, J. A., Rowland, D. L., and Webster, T. M.
(2015). Palmer {Amaranth} ({Amaranthus} palmeri) {Competition} for
{Water} in {Cotton}. \emph{Weed Science} 63, 928--935.
doi:\href{https://doi.org/10.1614/WS-D-15-00062.1}{10.1614/WS-D-15-00062.1}.

\leavevmode\hypertarget{ref-bravo2017}{}%
Bravo, W., Leon, R. G., Ferrell, J. A., Mulvaney, M. J., and Wood, C. W.
(2017). Differentiation of {Life}-{History Traits} among {Palmer
Amaranth Populations} ({Amaranthus} palmeri) and {Its Relation} to
{Cropping Systems} and {Glyphosate Sensitivity}. \emph{Weed Science} 65,
339--349.
doi:\href{https://doi.org/10.1017/wsc.2017.14}{10.1017/wsc.2017.14}.

\leavevmode\hypertarget{ref-bravo2018}{}%
Bravo, W., Leon, R. G., Ferrell, J. A., Mulvaney, M. J., and Wood, C. W.
(2018). Evolutionary {Adaptations} of {Palmer Amaranth} ({Amaranthus}
palmeri) to {Nitrogen Fertilization} and {Crop Rotation History Affect
Morphology} and {Nutrient}-{Use Efficiency}. \emph{Weed Science} 66,
180--189.
doi:\href{https://doi.org/10.1017/wsc.2017.73}{10.1017/wsc.2017.73}.

\leavevmode\hypertarget{ref-briscoerunquist2019}{}%
Briscoe Runquist, R. D., Lake, T., Tiffin, P., and Moeller, D. A.
(2019). Species distribution models throughout the invasion history of
{Palmer} amaranth predict regions at risk of future invasion and reveal
challenges with modeling rapidly shifting geographic ranges. \emph{Sci
Rep} 9, 2426.
doi:\href{https://doi.org/10.1038/s41598-018-38054-9}{10.1038/s41598-018-38054-9}.

\leavevmode\hypertarget{ref-chahal2021}{}%
Chahal, P. S., Barnes, E. R., and Jhala, A. J. (2021). Emergence pattern
of {Palmer} amaranth ({Amaranthus} palmeri) influenced by tillage
timings and residual herbicides. \emph{Weed Technology} 35, 433--439.
doi:\href{https://doi.org/10.1017/wet.2020.136}{10.1017/wet.2020.136}.

\leavevmode\hypertarget{ref-chahal2018}{}%
Chahal, P. S., Irmak, S., Jugulam, M., and Jhala, A. J. (2018).
Evaluating {Effect} of {Degree} of {Water Stress} on {Growth} and
{Fecundity} of {Palmer} amaranth ({Amaranthus} palmeri) {Using Soil
Moisture Sensors}. \emph{Weed Science} 66, 738--745.
doi:\href{https://doi.org/10.1017/wsc.2018.47}{10.1017/wsc.2018.47}.

\leavevmode\hypertarget{ref-correndo2021}{}%
Correndo, A. A., Moro Rosso, L. H., and Ciampitti, I. A. (2021).
Retrieving and processing agro-meteorological data from {API}-client
sources using {R} software. \emph{BMC Research Notes} 14, 205.
doi:\href{https://doi.org/10.1186/s13104-021-05622-8}{10.1186/s13104-021-05622-8}.

\leavevmode\hypertarget{ref-darwin1888}{}%
Darwin, C. (1888). \emph{The {Different Forms} of {Flowers} on {Plants}
of the {Same Species}}. {J. Murray} Available at:
\url{http://books.google.com?id=7uMEAAAAYAAJ}.

\leavevmode\hypertarget{ref-davis2015}{}%
Davis, A. S., Schutte, B. J., Hager, A. G., and Young, B. G. (2015).
Palmer {Amaranth} ({Amaranthus} palmeri) {Damage Niche} in {Illinois
Soybean Is Seed Limited}. \emph{Weed Science} 63, 658--668.
doi:\href{https://doi.org/10.1614/WS-D-14-00177.1}{10.1614/WS-D-14-00177.1}.

\leavevmode\hypertarget{ref-farmer2017}{}%
Farmer, J. A., Webb, E. B., Pierce, R. A., and Bradley, K. W. (2017).
Evaluating the potential for weed seed dispersal based on waterfowl
consumption and seed viability. \emph{Pest Management Science} 73,
2592--2603. doi:\href{https://doi.org/10.1002/ps.4710}{10.1002/ps.4710}.

\leavevmode\hypertarget{ref-fisher1930}{}%
Fisher, R. A. (1930). The genetical theory of natural selection.
\emph{Eugen Rev} 22, 127--130. Available at:
\url{https://www.ncbi.nlm.nih.gov/pmc/articles/PMC2984947/} {[}Accessed
August 12, 2021{]}.

\leavevmode\hypertarget{ref-fox2018}{}%
Fox, J., and Weisberg, S. (2018). \emph{An {R Companion} to {Applied
Regression}}. {SAGE Publications} Available at:
\url{http://books.google.com?id=uPNrDwAAQBAJ}.

\leavevmode\hypertarget{ref-garetson2019}{}%
Garetson, R., Singh, V., Singh, S., Dotray, P., and Bagavathiannan, M.
(2019). Distribution of herbicide-resistant {Palmer} amaranth
({Amaranthus} palmeri) in row crop production systems in {Texas}.
\emph{Weed Technology} 33, 355--365.
doi:\href{https://doi.org/10.1017/wet.2019.14}{10.1017/wet.2019.14}.

\leavevmode\hypertarget{ref-guo2003}{}%
Guo, P., and Al-Khatib, K. (2003). Temperature effects on germination
and growth of redroot pigweed ({Amaranthus} retroflexus), {Palmer}
amaranth ({A}. Palmeri), and common waterhemp ({A}. rudis). \emph{Weed
Science} 51, 869--875.
doi:\href{https://doi.org/10.1614/P2002-127}{10.1614/P2002-127}.

\leavevmode\hypertarget{ref-hartzler2016}{}%
Hartzler, B., and Anderson, M. (2016). Palmer amaranth: {It}'s here, now
what? 10.

\leavevmode\hypertarget{ref-heap2021}{}%
Heap, I. (2021). Internation {Herbicide}-{Resistant Weed Database}.
Available at: \url{http://www.weedscience.org/Home.aspx} {[}Accessed
July 26, 2021{]}.

\leavevmode\hypertarget{ref-hothorn2008}{}%
Hothorn, T., Bretz, F., and Westfall, P. (2008). Simultaneous
{Inference} in {General Parametric Models}. \emph{Biometrical Journal}
50, 346--363.
doi:\href{https://doi.org/10.1002/bimj.200810425}{10.1002/bimj.200810425}.

\leavevmode\hypertarget{ref-jha2010}{}%
Jha, P., Norsworthy, J. K., Riley, M. B., and Bridges, W. (2010). Annual
{Changes} in {Temperature} and {Light Requirements} for {Germination} of
{Palmer Amaranth} ({Amaranthus} palmeri) {Seeds Retrieved} from {Soil}.
\emph{Weed Science} 58, 426--432.
doi:\href{https://doi.org/10.1614/WS-D-09-00038.1}{10.1614/WS-D-09-00038.1}.

\leavevmode\hypertarget{ref-jhala2021}{}%
Jhala, A. J., Norsworthy, J. K., Ganie, Z. A., Sosnoskie, L. M., Beckie,
H. J., Mallory-Smith, C. A., Liu, J., Wei, W., Wang, J., and
Stoltenberg, D. E. (2021). Pollen-mediated gene flow and transfer of
resistance alleles from herbicide-resistant broadleaf weeds. \emph{Weed
Technology} 35, 173--187.
doi:\href{https://doi.org/10.1017/wet.2020.101}{10.1017/wet.2020.101}.

\leavevmode\hypertarget{ref-keeley1987}{}%
Keeley, P. E., Carter, C. H., and Thullen, R. J. (1987). Influence of
{Planting Date} on {Growth} of {Palmer Amaranth} ({Amaranthus} palmeri).
\emph{Weed Science} 35, 199--204.
doi:\href{https://doi.org/10.1017/S0043174500079054}{10.1017/S0043174500079054}.

\leavevmode\hypertarget{ref-kistner2018}{}%
Kistner, E. J., and Hatfield, J. L. (2018). Potential {Geographic
Distribution} of {Palmer Amaranth} under {Current} and {Future
Climates}. \emph{Agricultural \& Environmental Letters} 3, 170044.
doi:\href{https://doi.org/10.2134/ael2017.12.0044}{10.2134/ael2017.12.0044}.

\leavevmode\hypertarget{ref-klingaman1994}{}%
Klingaman, T. E., and Oliver, L. R. (1994). Palmer {Amaranth}
({Amaranthus} palmeri) {Interference} in {Soybeans} ({Glycine} max).
\emph{Weed Science} 42, 523--527.
doi:\href{https://doi.org/10.1017/S0043174500076888}{10.1017/S0043174500076888}.

\leavevmode\hypertarget{ref-kohrt2017}{}%
Kohrt, J. R., Sprague, C. L., Nadakuduti, S. S., and Douches, D. (2017).
Confirmation of a {Three}-{Way} ({Glyphosate}, {ALS}, and {Atrazine})
{Herbicide}-{Resistant Population} of {Palmer Amaranth} ({Amaranthus}
palmeri) in {Michigan}. \emph{Weed Science} 65, 327--338.
doi:\href{https://doi.org/10.1017/wsc.2017.2}{10.1017/wsc.2017.2}.

\leavevmode\hypertarget{ref-korres2017a}{}%
Korres, N. E., and Norsworthy, J. K. (2017). Palmer {Amaranth}
({Amaranthus} palmeri) {Demographic} and {Biological Characteristics} in
{Wide}-{Row Soybean}. \emph{Weed Science} 65, 491--503.
doi:\href{https://doi.org/10.1017/wsc.2017.12}{10.1017/wsc.2017.12}.

\leavevmode\hypertarget{ref-korres2017}{}%
Korres, N. E., Norsworthy, J. K., FitzSimons, T., Roberts, T. L., and
Oosterhuis, D. M. (2017). Differential {Response} of {Palmer Amaranth}
({Amaranthus} palmeri) {Gender} to {Abiotic Stress}. \emph{Weed Science}
65, 213--227.
doi:\href{https://doi.org/10.1017/wsc.2016.34}{10.1017/wsc.2016.34}.

\leavevmode\hypertarget{ref-korres2020}{}%
Korres, N. E., Norsworthy, J. K., Mauromoustakos, A., and Williams, M.
M. (2020). Soybean density and {Palmer} amaranth ({Amaranthus} palmeri)
establishment time: Effects on weed biology, crop yield, and economic
returns. \emph{Weed Science} 68, 467--475.
doi:\href{https://doi.org/10.1017/wsc.2020.41}{10.1017/wsc.2020.41}.

\leavevmode\hypertarget{ref-kupper2017}{}%
Küpper, A., Borgato, E. A., Patterson, E. L., Netto, A. G., Nicolai, M.,
Carvalho, S. J. P. de, Nissen, S. J., Gaines, T. A., and Christoffoleti,
P. J. (2017). Multiple {Resistance} to {Glyphosate} and {Acetolactate
Synthase Inhibitors} in {Palmer Amaranth} ({Amaranthus} palmeri)
{Identified} in {Brazil}. \emph{Weed Science} 65, 317--326.
doi:\href{https://doi.org/10.1017/wsc.2017.1}{10.1017/wsc.2017.1}.

\leavevmode\hypertarget{ref-lang1965}{}%
Lang, A. (1965). {``Physiology of flower initiation,''} in
\emph{Differenzierung und {Entwicklung} / {Differentiation} and
{Development}} Handbuch der {Pflanzenphysiologie} / {Encyclopedia} of
{Plant Physiology}., ed. A. Lang ({Berlin, Heidelberg}: {Springer}),
1380--1536.
doi:\href{https://doi.org/10.1007/978-3-642-50088-6_39}{10.1007/978-3-642-50088-6\_39}.

\leavevmode\hypertarget{ref-larran2017}{}%
Larran, A. S., Palmieri, V. E., Perotti, V. E., Lieber, L., Tuesca, D.,
and Permingeat, H. R. (2017). Target-site resistance to acetolactate
synthase ({ALS})-inhibiting herbicides in {Amaranthus} palmeri from
{Argentina}. \emph{Pest Management Science} 73, 2578--2584.
doi:\href{https://doi.org/10.1002/ps.4662}{10.1002/ps.4662}.

\leavevmode\hypertarget{ref-leeper2017}{}%
Leeper, T. J. (2017). Interpreting {Regression Results} using {Average
Marginal Effects} with {R}'s margins. 31.

\leavevmode\hypertarget{ref-lindsay2017}{}%
Lindsay, K., Popp, M., Norsworthy, J., Bagavathiannan, M., Powles, S.,
and Lacoste, M. (2017). {PAM}: {Decision Support} for {Long}-{Term
Palmer Amaranth} ({Amaranthus} palmeri) {Control}. \emph{Weed
Technology} 31, 915--927.
doi:\href{https://doi.org/10.1017/wet.2017.69}{10.1017/wet.2017.69}.

\leavevmode\hypertarget{ref-maclaren2020}{}%
MacLaren, C., Storkey, J., Menegat, A., Metcalfe, H., and
Dehnen-Schmutz, K. (2020). An ecological future for weed science to
sustain crop production and the environment. {A} review. \emph{Agron.
Sustain. Dev.} 40, 24.
doi:\href{https://doi.org/10.1007/s13593-020-00631-6}{10.1007/s13593-020-00631-6}.

\leavevmode\hypertarget{ref-macrae2013}{}%
MacRae, A. W., Webster, T. M., Sosnoskie, L. M., Culpepper, A. S., and
Kichler, J. M. (2013). Cotton {Yield Loss Potential} in {Response} to
{Length} of {Palmer Amaranth} ({Amaranthus} palmeri) {Interference}. 17,
6.

\leavevmode\hypertarget{ref-massinga2001}{}%
Massinga, R. A., Currie, R. S., Horak, M. J., and Boyer, J. (2001).
Interference of {Palmer} amaranth in corn. \emph{Weed Science} 49,
202--208.
doi:\href{https://doi.org/10.1614/0043-1745(2001)049\%5B0202:IOPAIC\%5D2.0.CO;2}{10.1614/0043-1745(2001)049{[}0202:IOPAIC{]}2.0.CO;2}.

\leavevmode\hypertarget{ref-matzrafi2021}{}%
Matzrafi, M., Osipitan, O. A., Ohadi, S., and Mesgaran, M. B. (2021).
Under pressure: Maternal effects promote drought tolerance in progeny
seed of {Palmer} amaranth ({Amaranthus} palmeri). \emph{Weed Science}
69, 31--38.
doi:\href{https://doi.org/10.1017/wsc.2020.75}{10.1017/wsc.2020.75}.

\leavevmode\hypertarget{ref-mcfarlane2018}{}%
McFarlane, G. R., Whitelaw, C. B. A., and Lillico, S. G. (2018).
{CRISPR}-{Based Gene Drives} for {Pest Control}. \emph{Trends in
Biotechnology} 36, 130--133.
doi:\href{https://doi.org/10.1016/j.tibtech.2017.10.001}{10.1016/j.tibtech.2017.10.001}.

\leavevmode\hypertarget{ref-menges1987}{}%
Menges, R. M. (1987). Weed {Seed Population Dynamics} during {Six Years}
of {Weed Management Systems} in {Crop Rotations} on {Irrigated Soil}.
\emph{Weed Science} 35, 328--332. Available at:
\url{http://www.jstor.org/stable/4044593}.

\leavevmode\hypertarget{ref-mesgaran2021}{}%
Mesgaran, M. B., Matzrafi, M., and Ohadi, S. (2021). Sex dimorphism in
dioecious {Palmer} amaranth ({Amaranthus} palmeri) in response to water
stress. \emph{Planta} 254, 17.
doi:\href{https://doi.org/10.1007/s00425-021-03664-7}{10.1007/s00425-021-03664-7}.

\leavevmode\hypertarget{ref-milani2021}{}%
Milani, A., Panozzo, S., Farinati, S., Iamonico, D., Sattin, M., Loddo,
D., and Scarabel, L. (2021). Recent {Discovery} of {Amaranthus} palmeri
{S}. {Watson} in {Italy}: {Characterization} of {ALS}-{Resistant
Populations} and {Sensitivity} to {Alternative Herbicides}.
\emph{Sustainability} 13, 7003.
doi:\href{https://doi.org/10.3390/su13137003}{10.3390/su13137003}.

\leavevmode\hypertarget{ref-montgomery2021}{}%
Montgomery, J. S., Giacomini, D. A., Weigel, D., and Tranel, P. J.
(2021). Male-specific {Y}-chromosomal regions in waterhemp ({Amaranthus}
tuberculatus) and {Palmer} amaranth ({Amaranthus} palmeri). \emph{New
Phytologist} 229, 3522--3533.
doi:\href{https://doi.org/10.1111/nph.17108}{10.1111/nph.17108}.

\leavevmode\hypertarget{ref-montgomery2019}{}%
Montgomery, J. S., Sadeque, A., Giacomini, D. A., Brown, P. J., and
Tranel, P. J. (2019). Sex-specific markers for waterhemp ({Amaranthus}
tuberculatus) and {Palmer} amaranth ({Amaranthus} palmeri). \emph{Weed
Science} 67, 412--418.
doi:\href{https://doi.org/10.1017/wsc.2019.27}{10.1017/wsc.2019.27}.

\leavevmode\hypertarget{ref-morgan2001}{}%
Morgan, G. D., Baumann, P. A., and Chandler, J. M. (2001). Competitive
{Impact} of {Palmer Amaranth} ({Amaranthus} palmeri) on {Cotton}
({Gossypium} hirsutum) {Development} and {Yield}. \emph{Weed Technology}
15, 408--412.
doi:\href{https://doi.org/10.1614/0890-037X(2001)015\%5B0408:CIOPAA\%5D2.0.CO;2}{10.1614/0890-037X(2001)015{[}0408:CIOPAA{]}2.0.CO;2}.

\leavevmode\hypertarget{ref-nguyen2018}{}%
Nguyen, H. M., Kleitou, P., Kletou, D., Sapir, Y., and Winters, G.
(2018). Differences in flowering sex ratios between native and invasive
populations of the seagrass {Halophila} stipulacea. \emph{Botanica
Marina} 61, 337--342.
doi:\href{https://doi.org/10.1515/bot-2018-0015}{10.1515/bot-2018-0015}.

\leavevmode\hypertarget{ref-oliveira2018}{}%
Oliveira, M. C., Gaines, T. A., Patterson, E. L., Jhala, A. J., Irmak,
S., Amundsen, K., and Knezevic, S. Z. (2018). Interspecific and
intraspecific transference of metabolism-based mesotrione resistance in
dioecious weedy {Amaranthus}. \emph{The Plant Journal} 96, 1051--1063.
doi:\href{https://doi.org/10.1111/tpj.14089}{10.1111/tpj.14089}.

\leavevmode\hypertarget{ref-oliveira2021a}{}%
Oliveira, M. C., Giacomini, D. A., Arsenijevic, N., Vieira, G., Tranel,
P. J., and Werle, R. (2021). Distribution and validation of genotypic
and phenotypic glyphosate and {PPO}-inhibitor resistance in {Palmer}
amaranth ({Amaranthus} palmeri) from southwestern {Nebraska}. \emph{Weed
Technology} 35, 65--76.
doi:\href{https://doi.org/10.1017/wet.2020.74}{10.1017/wet.2020.74}.

\leavevmode\hypertarget{ref-page2021}{}%
Page, E. R., Nurse, R. E., Meloche, S., Bosveld, K., Grainger, C.,
Obeid, K., Filotas, M., Simard, M.-J., and Laforest, M. (2021). Import
of {Palmer} amaranth ({Amaranthus} palmeri {S}. {Wats}.) Seed with sweet
potato ({Ipomea} batatas ({L}.) {Lam}) slips. \emph{Can. J. Plant Sci.},
CJPS-2020-0321.
doi:\href{https://doi.org/10.1139/CJPS-2020-0321}{10.1139/CJPS-2020-0321}.

\leavevmode\hypertarget{ref-perez-llorca2019}{}%
Pérez-Llorca, M., and Sánchez Vilas, J. (2019). Sexual dimorphism in
response to herbivory and competition in the dioecious herb {Spinacia}
oleracea. \emph{Plant Ecol} 220, 57--68.
doi:\href{https://doi.org/10.1007/s11258-018-0902-7}{10.1007/s11258-018-0902-7}.

\leavevmode\hypertarget{ref-piskackova2021}{}%
Piskackova, T. A. R., Reberg-Horton, S. C., Richardson, R. J., Jennings,
K. M., Franca, L., Young, B. G., and Leon, R. G. (2021). Windows of
action for controlling palmer amaranth ({Amaranthus} palmeri) using
emergence and phenology models. \emph{Weed Research} 61, 188--198.
doi:\href{https://doi.org/10.1111/wre.12470}{10.1111/wre.12470}.

\leavevmode\hypertarget{ref-price2011}{}%
Price, A. J., Balkcom, K. S., Culpepper, S. A., Kelton, J. A., Nichols,
R. L., and Schomberg, H. (2011). Glyphosate-resistant {Palmer} amaranth:
{A} threat to conservation tillage. \emph{Journal of Soil and Water
Conservation} 66, 265--275.
doi:\href{https://doi.org/10.2489/jswc.66.4.265}{10.2489/jswc.66.4.265}.

\leavevmode\hypertarget{ref-ritz2015}{}%
Ritz, C., Baty, F., Streibig, J. C., and Gerhard, D. (2015).
Dose-{Response Analysis Using R}. \emph{PLOS ONE} 10, e0146021.
doi:\href{https://doi.org/10.1371/journal.pone.0146021}{10.1371/journal.pone.0146021}.

\leavevmode\hypertarget{ref-rumpa2019}{}%
Rumpa, M. M., Krausz, R. F., Gibson, D. J., and Gage, K. L. (2019).
Effect of {PPO}-{Inhibiting Herbicides} on the {Growth} and {Sex Ratio}
of a {Dioecious Weed Species Amaranthus} palmeri ({Palmer Amaranth}).
\emph{Agronomy} 9, 275.
doi:\href{https://doi.org/10.3390/agronomy9060275}{10.3390/agronomy9060275}.

\leavevmode\hypertarget{ref-sanctis2021}{}%
Sanctis, J. H. S. de, Barnes, E. R., Knezevic, S. Z., Kumar, V., and
Jhala, A. J. (2021). Residual herbicides affect critical time of
{Palmer} amaranth removal in soybean. \emph{Agronomy Journal} 113,
1920--1933.
doi:\href{https://doi.org/10.1002/agj2.20615}{10.1002/agj2.20615}.

\leavevmode\hypertarget{ref-sauer1957}{}%
Sauer, J. (1957). Recent {Migration} and {Evolution} of the {Dioecious
Amaranths}. \emph{Evolution} 11, 11--31.
doi:\href{https://doi.org/10.2307/2405808}{10.2307/2405808}.

\leavevmode\hypertarget{ref-sauer1972}{}%
Sauer, J. D. (1972). The dioecious amaranths: A new species name and
major range extensions. \emph{Madroño} 21, 426--434. Available at:
\url{http://www.jstor.org/stable/41423815}.

\leavevmode\hypertarget{ref-sbatella2010}{}%
Sbatella, G. M., and Wilson, R. G. (2010). Isoxaflutole {Shifts Kochia}
({Kochia} scoparia) {Populations} in {Continuous Corn}. \emph{Weed
Technology} 24, 392--396.
doi:\href{https://doi.org/10.1614/WT-D-09-00023.1}{10.1614/WT-D-09-00023.1}.

\leavevmode\hypertarget{ref-schliekelman2005}{}%
Schliekelman, P., Ellner, S., and Gould, F. (2005). Pest {Control} by
{Genetic Manipulation} of {Sex Ratio}. \emph{Journal of Economic
Entomology} 98, 18--34.
doi:\href{https://doi.org/10.1093/jee/98.1.18}{10.1093/jee/98.1.18}.

\leavevmode\hypertarget{ref-schwartz2016}{}%
Schwartz, L. M., Norsworthy, J. K., Young, B. G., Bradley, K. W.,
Kruger, G. R., Davis, V. M., Steckel, L. E., and Walsh, M. J. (2016).
Tall {Waterhemp} ({Amaranthus} tuberculatus) and {Palmer} amaranth
({Amaranthus} palmeri) {Seed Production} and {Retention} at {Soybean
Maturity}. \emph{Weed Technology} 30, 284--290.
doi:\href{https://doi.org/10.1614/WT-D-15-00130.1}{10.1614/WT-D-15-00130.1}.

\leavevmode\hypertarget{ref-scott2011}{}%
Scott, D. (2011). The {Technological Fix Criticisms} and the
{Agricultural Biotechnology Debate}. \emph{J Agric Environ Ethics} 24,
207--226.
doi:\href{https://doi.org/10.1007/s10806-010-9253-7}{10.1007/s10806-010-9253-7}.

\leavevmode\hypertarget{ref-sosnoskie2013}{}%
Sosnoskie, L. M., Webster, T. M., and Culpepper, A. S. (2013).
Glyphosate {Resistance Does Not Affect Palmer Amaranth} ({Amaranthus}
palmeri) {Seedbank Longevity}. \emph{Weed Science} 61, 283--288.
doi:\href{https://doi.org/10.1614/WS-D-12-00111.1}{10.1614/WS-D-12-00111.1}.

\leavevmode\hypertarget{ref-spaunhorst2018}{}%
Spaunhorst, D. J., Devkota, P., Johnson, W. G., Smeda, R. J., Meyer, C.
J., and Norsworthy, J. K. (2018). Phenology of {Five Palmer} amaranth
({Amaranthus} palmeri) {Populations Grown} in {Northern Indiana} and
{Arkansas}. \emph{Weed Science} 66, 457--469.
doi:\href{https://doi.org/10.1017/wsc.2018.12}{10.1017/wsc.2018.12}.

\leavevmode\hypertarget{ref-vanwychen2020}{}%
Van Wychen, L. (2020). 2020 {Survey} of the most common and troublesome
weeds in grass crops, pasture, and turf in the {United States} and
{Canada}. Available at:
\url{https://wssa.net/wp-content/uploads/2020-Weed-Survey_grass-crops.xlsx}.

\leavevmode\hypertarget{ref-wang1992}{}%
Wang, J. L., Klessig, D. F., and Berry, J. O. (1992). Regulation of {C4
Gene Expression} in {Developing Amaranth Leaves}. \emph{The Plant Cell}
4, 173--184.
doi:\href{https://doi.org/10.1105/tpc.4.2.173}{10.1105/tpc.4.2.173}.

\leavevmode\hypertarget{ref-ward2013}{}%
Ward, S. M., Webster, T. M., and Steckel, L. E. (2013). Palmer
{Amaranth} ({Amaranthus} palmeri): {A Review}. \emph{Weed Technology}
27, 12--27.
doi:\href{https://doi.org/10.1614/WT-D-12-00113.1}{10.1614/WT-D-12-00113.1}.

\leavevmode\hypertarget{ref-webster2015}{}%
Webster, T. M., and Grey, T. L. (2015). Glyphosate-{Resistant Palmer
Amaranth} ({Amaranthus} palmeri) {Morphology}, {Growth}, and {Seed
Production} in {Georgia}. \emph{Weed Science} 63, 264--272.
doi:\href{https://doi.org/10.1614/WS-D-14-00051.1}{10.1614/WS-D-14-00051.1}.

\leavevmode\hypertarget{ref-yacouby2020}{}%
Yacouby, R., and Axman, D. (2020). Probabilistic {Extension} of
{Precision}, {Recall}, and {F1 Score} for {More Thorough Evaluation} of
{Classification Models}. in \emph{Proceedings of the {First Workshop} on
{Evaluation} and {Comparison} of {NLP Systems}} ({Online}: {Association
for Computational Linguistics}), 79--91.
doi:\href{https://doi.org/10.18653/v1/2020.eval4nlp-1.9}{10.18653/v1/2020.eval4nlp-1.9}.

\leavevmode\hypertarget{ref-yu2021}{}%
Yu, E., Blair, S., Hardel, M., Chandler, M., Thiede, D., Cortilet, A.,
Gunsolus, J., and Becker, R. (2021). Timeline of {Palmer} amaranth
({Amaranthus} palmeri) invasion and eradication in {Minnesota}.
\emph{Weed Technology}, 1--31.
doi:\href{https://doi.org/10.1017/wet.2021.32}{10.1017/wet.2021.32}.

\end{CSLReferences}

\hypertarget{figures}{%
\section*{Figures}\label{figures}}
\addcontentsline{toc}{section}{Figures}

\begin{figure}

{\centering \includegraphics[width=160mm,height=100mm]{../data analysis/weather/Figure 1} 

}

\caption{Mean average temperature (C) and total montly precipitation (mm) at Arlington, WI, Clay Center, NE, Grant, NE, Lincoln, NE and Macomb, IL}\label{fig:Figure-1}
\end{figure}

\begin{figure}

{\centering \includegraphics[width=150mm,height=70mm]{../data analysis/figures/Figure 2} 

}

\caption{Palmer amaranth adaptation study layout of a plant cohort timing. Twenty-four Palmer amaranth plants were place 76.2 cm apart in each experimental unit}\label{fig:Figure-2}
\end{figure}

\begin{figure}

{\centering \includegraphics[width=150mm,height=90mm]{../data analysis/figures/Figure 3} 

}

\caption{Palmer amaranth biomass (A) and height (B) growing in corn, bareground, and soybean across Arlington, WI, Clay Center, NE, Grant, NE, Lincoln, NE and Macomb, IL}\label{fig:Figure-3}
\end{figure}

\begin{figure}

{\centering \includegraphics[width=150mm,height=150mm]{../data analysis/figures/Figure 4} 

}

\caption{Cumulative flowering of Palmer amaranth at first and second transplant timing (A) and day of year of 10, 50, and 90 cumulative flowering at first and second transplant timing (B)}\label{fig:Figure-4}
\end{figure}

\begin{figure}

{\centering \includegraphics[width=170mm,height=70mm]{../data analysis/figures/Figure 5} 

}

\caption{Cumulative flowering of Palmer amaranth at first and second transplant timing (A) and day of year of 10, 50, and 90 cumulative flowering at first and second transplant timing (B)}\label{fig:Figure-5}
\end{figure}

\begin{figure}

{\centering \includegraphics[width=150mm,height=90mm]{../data analysis/figures/Figure 6} 

}

\caption{MCO (180 cm) holding harvested Palmer amaranth plants at 40 days after first transplant (1st cohort, A) and 33 days after second transplant (2nd cohort, B). From left to right, Palmer amaranth growing in bareground, soybean and corn in Arlington, Wisconsin}\label{fig:Figure-6}
\end{figure}

\end{document}
