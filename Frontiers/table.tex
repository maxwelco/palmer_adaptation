\begin{table}[!h]

\caption{\label{tab:unnamed-chunk-2}Field study attributes }
\centering
\resizebox{\linewidth}{!}{
\fontsize{10}{12}\selectfont
\begin{tabular}[t]{lllllll}
\toprule
Attributes &  & Arlington, WI & Clay Center, NE & Grant, NE & Lincoln, NE & Macomb, IL\\
\midrule
Bareground &  &  &  &  &  & \\

 & Weed control & glyphosate & saflufenacil + imazethapyr + pyroxasulfone &  &  & \\

Corn &  &  &  &  &  & \\

 & Hybrid & NK0142 3120-EZ1 & DKC60-67 &  &  & \\

 & Seeding rate & 88956 & 86487 &  &  & \\

 & Weed control & glyphosate\textsuperscript{a} / \emph{S}-metolachor\textsuperscript{b} & \emph{S}-metolachlor + trazine + mesotrione + bicyclopyrone\textsuperscript{c} &  &  & \\

 & Stage at 1\textsuperscript{st} cohort & V2-3 &  &  &  & \\

 & Stage at 2\textsuperscript{nd} cohort & V6-7 &  &  &  & \\

 & Planting day & April 30, 2018 / May 5, 2019 & May 10, 2018/19 &  &  & \\

\multirow{-7}{*}{\raggedright\arraybackslash } & Fertilization & N (46-0-0) at 157 kg ha\textsuperscript{-1} &  &  &  & \\

Soybean &  &  &  &  &  & \\

 & Variety & DSR-1950 & AG21X8 &  &  & \\

 & Seeding rate & 296400 & 321237 &  &  & \\

 & Weed control & glyphosate / \emph{S}-metolachor & saflufenacil + imazethapyr\textsuperscript{d} + pyroxasulfone &  &  & \\

 & Stage at 1\textsuperscript{st} cohort & V1-2 &  &  &  & \\

 & Stage at 2\textsuperscript{nd} cohort & V5-6 &  &  &  & \\

\multirow{-6}{*}{\raggedright\arraybackslash } & Planting day & May 5, 2018 / May 10, 2009 & May 14, 2018/19 &  &  & \\

Soil &  &  &  &  &  & \\

 & Type & Plano-silt-loam & Crete Silt Loam &  &  & \\

 & Ratio (sand-clay-silt) & 10-64-26 & 17-58-25 &  &  & \\

 & pH & 6.6 & 6.5 &  &  & \\

\multirow{-4}{*}{\raggedright\arraybackslash } & Organic matter (\%) & 3.5 & 3 &  &  & \\
\bottomrule
\multicolumn{7}{l}{\rule{0pt}{1em}\textsuperscript{a} glyphoste, 840 g ae ha \textsuperscript{b} S-metolachor, 1324 g ai ha; \textsuperscript{c} S-metolachlor + trazine + mesotrione, + bicyclopyrone, 2409 g ai ha; \textsuperscript{d} saflufenacil + imazethapyr + pyroxasulfone, 215 g ai ha}\\
\end{tabular}}
\end{table}





\begin{table}[!h]

\caption{\label{tab:unnamed-chunk-3}Average marginal means of Palmer amaranth sex dimorphism logistic model. Factor pararemter values (crop and bareground) is shown compared to soybean.}
\centering
\fontsize{10}{12}\selectfont
\begin{tabular}[t]{lrrrrrr}
\toprule
Term & AME\textsuperscript{a} & SE\textsuperscript{b} & Lower & Upper & Z-score & P-value\\
\midrule
crop\_bareground & -0.048 & 0.054 & -0.154 & 0.059 & -0.876 & 0.381\\

crop\_corn & -0.148 & 0.052 & -0.250 & -0.046 & -2.842 & 0.004\\

doyh & 0.004 & 0.001 & 0.003 & 0.006 & 4.959 & 0.000\\

height & 0.002 & 0.001 & 0.001 & 0.003 & 2.953 & 0.003\\

weight & 0.001 & 0.000 & 0.000 & 0.001 & 2.179 & 0.029\\
\bottomrule
\multicolumn{7}{l}{\rule{0pt}{1em}\textsuperscript{a} Average Marginal Effects. \textsuperscript{b} Standard Error.}\\
\end{tabular}
\end{table}